\section{预备知识}

\subsection{随机过程的概率分布}

\begin{frame}
  \frametitle{有限维分布和数字特征}
  对于随机过程$\{X(t),t\in T\}$,过程的一维分布为
  $$
    F_t(x)=P(X(t)\leqslant x),
  $$
  过程的\textbf{一维均值函数}为
  $$
    \mu_X(t)=EX(t),
  $$
  过程的\textbf{方差函数}为
  $$
    \sigma_X^2(t)=\mathrm{Var}(X(t)).
  $$
\end{frame}

\begin{frame}
  \frametitle{有限维分布和数字特征}
  过程在$t_1,t_2$两个不同时刻值的联合二维分布为
  $$
  F_{t_1,t_2}(x_1,x_2)=P(X(t_1)\leqslant x_1,X(t_2)\leqslant x_2),
  $$
  过程的\textbf{自相关函数}为
  $$
  r_X(t_1,t_2)=E(X_1(t)X_2(t)),
  $$
  过程的\textbf{协方差函数}为
  $$
  R_X(t_1,t_2)\equiv\mathrm{Cov}(X(t_1),X(t_2))=E[(X(t_1)-\mu_X(t_1))(X(t_2)-\mu_X(t_2))].
  $$
\end{frame}

\begin{frame}
  \frametitle{自相关函数与协方差函数的性质}
  \begin{enumerate}
    \item \textbf{对称性}. 对任何$s,t$,有
          \begin{align*}
            r_X(s,t)&=r_X(t,s),\\
            R_X(s,t)&=R_X(t,s).
          \end{align*}

    \item \textbf{非负定性}. 对任何$t_1,t_2,\cdots,t_n\in T$及任意系数$b_1,b_2,\cdots,b_n$有
          \begin{align*}
            \sum_{i=1}^n\sum_{j=1}^nb_ib_jr_X(t_i,t_j)&\geqslant 0,\\
            \sum_{i=1}^n\sum_{j=1}^nb_ib_jR_X(t_i,t_j)&\geqslant 0.\\
          \end{align*}

  \end{enumerate}

\end{frame}

\subsection{平稳过程与独立增量过程}

\begin{frame}
    \frametitle{严平稳过程}
    \begin{mydefinition}[同分布的随机向量]
        如果一个随机向量$\bm X=(X_1,X_2,\cdots,X_n)$与
        另一个随机向量$\bm Y=(Y_1,Y_2,\cdots,Y_n)$有
        相同的联合分布函数,则称这两个随机向量是同分布的,
        记为
        $$
          \bm X\overset{d}{=}\bm Y.
        $$        
    \end{mydefinition}

    \begin{mydefinition}[严平稳过程]\label{def:strictly-stationary-process}
        如果随机过程$X(t)$对任意的$t_1,\cdots,t_n\in T$
        和任何$h$都有
        \begin{equation}\label{eq:strictly-stationary}
            (X(t_1+h),\cdots,X(t_n+h))\overset{d}{=} (X(t_1),\cdots,X(t_n)),            
        \end{equation}

        则称$X(t)$为\textbf{严格平稳的}(strictly stationary or strict-sense stationary). 
    \end{mydefinition}

\end{frame}

\begin{frame}
    \frametitle{(宽)平稳过程}
    \begin{mydefinition}[宽平稳过程]\label{def:wide-stationary-process}
        如果随机过程$X(t)$的所有二阶矩存在,并且$EX(t)=m$为常数,
        协方差函数$R_X(s,t)$只与时间差$t-s$有关,
        则称$X(t)$为\textbf{(宽)平稳的}(wide-sense stationary)或\textbf{二阶矩平稳的}. 
    \end{mydefinition}
    对宽平稳过程$X(t)$,由于对$-\infty<s,t<+\infty$,
    $$
        R_X(s,t)=R_X(0,t-s),
    $$
    所以可以记为$R_X(t-s)$. 显然对所有$t$,都有$R_X(t)=R_X(-t)$,
    即$R_X(t)$为偶函数. 

\end{frame}

\begin{frame}\label{frame:wide-stationary-strict-stationary}
    \frametitle{宽平稳过程与严平稳过程之间的关系}
    \begin{alertblock}{注意}
        一般来说,这两个过程是互不包含的.

        \begin{itemize}
            \item 严平稳过程由于不一定有二阶矩而不一定是宽平稳的;
            \item 宽平稳过程由于其有限维联合分布可能不满足式(\ref{eq:strictly-stationary})
                    而不一定是严平稳的. 
        \end{itemize}
        
        
    \end{alertblock}
    \begin{exampleblock}{特例}
        \begin{itemize}
            \item 如果严平稳过程的二阶矩存在,则它是宽平稳的;
            \item 如果宽平稳过程是Gauss过程,则它是严平稳的. 
        \end{itemize}
    \end{exampleblock}
\end{frame}

\begin{frame}
    \frametitle{独立增量过程}
    \begin{mydefinition}[独立增量过程与平稳独立增量过程]\label{def:independent-increment-process}
        对任意的$t_1<t_2<\cdots<t_n$且
        $t_1,\cdots,t_n\in T$,如果
        随机变量
        $$
            X(t_2)-X(t_1),X(t_3)-X(t_2),\cdots,X(t_n)-X(t_{n-1})
        $$
        是相互独立的,则称$X(t)$为\textbf{独立增量过程}.
        
        如果进一步有,对任意的$t_1,t_2$,
        \begin{equation*}
            X(t_1+h)-X(t_1)\overset{d}{=}X(t_2+h)-X(t_2),
        \end{equation*}
        则称$X(t)$为\textbf{平稳独立增量过程}. 
    \end{mydefinition}
    \begin{myproposition}
        可以证明,平稳独立增量过程的均值函数一定是$t$的线性函数. 
    \end{myproposition}
\end{frame}

\subsection{条件期望与矩母函数}

\begin{frame}
    \frametitle{条件期望}
    条件期望通常统一记为
    \begin{equation*}
        E(X|Y=y)=\int x\ \mathrm dF(x|y).
    \end{equation*}
    \begin{block}{注}
        $E(X|Y=y)=g(y)$ 是一个关于$y$的数值,
        而$E(X|Y)=g(Y)$ 是一个关于$Y$的随机变量函数,从而是随机变量.  
    \end{block}
\end{frame}

\begin{frame}
    \frametitle{条件期望的性质}
    \begin{myproposition}
        \begin{itemize}
            \item 若$X$与$Y$独立,则 $E(X|Y=y)=EX$.
            \item 条件期望的平滑性(全期望公式):
                \begin{equation}
                    E(E(X|Y))=EX.
                \end{equation}
            \item 对随机变量$X,Y$的函数$\phi(X,Y)$,有
                \begin{equation}
                    E[\phi(X,Y)|Y=y]=E[\phi(X,y)|Y=y].
                \end{equation}
        \end{itemize}
    \end{myproposition}
\end{frame}

\begin{frame}
    \frametitle{条件期望的性质}
    \begin{myproposition}
        若$X,Y$为随机变量,$EX,EY,E(g(Y))$存在,则
        \begin{itemize}
            \item 当$X,Y$独立时,有$E(Y|X)=EY$.
            \item $E(g(X)Y|X)=g(X)E(Y|X)$.
            \item $E(c|X)=c$.
            \item $E(g(X)|X)=g(X)$. 
            \item $E[Y-E(Y|X)]^2\leqslant E[Y-g(X)]^2$.
        \end{itemize}
    \end{myproposition}   
    \begin{block}{注}
        在均方误差最小的准则下,若要找到基于$X$对$Y$的最佳预报函数$g(\cdot)$,
        也就是要求出$g$使得$E[Y-g(X)]^2$最小,
        则可以证明所求出的$g(x)$就是$E(Y|X=x)$,这也就是最后一条性质的意义. 
    \end{block}
\end{frame}

\begin{frame}
    \frametitle{条件方差}
    \begin{mydefinition}[条件方差]
        若$E\{[Y-E(Y|X)]^2|X\}$存在,则称其为随机变量$X$条件下随机变量$Y$的
        条件方差,记为$D(Y|X)$.
    \end{mydefinition}
    \begin{myproposition}[条件方差的性质]
        \begin{itemize}
            \item $D(Y|X)=E\left(Y^2|X\right)-[E(Y|X)]^2$.
            \item $D(X|Y)=E\left(X^2|Y\right)-[E(X|Y)]^2$.
        \end{itemize}
    \end{myproposition}
    \begin{myproposition}
        \begin{equation*}
            D(Y)=E(D(Y|X))+D(E(Y|X)).
        \end{equation*}
    \end{myproposition}
\end{frame}

\begin{frame}
    \frametitle{矩母函数}
    \begin{mydefinition}[矩母函数]
        随机变量$X$的矩母函数(moment generating function, MGF)定义为
        随机变量$\exp\{tX\}$的期望,记作
        $g(t)$,即
        \begin{equation}
            g(t)=E\left[\mathrm e^{tX}\right].
        \end{equation}
    \end{mydefinition}
    \begin{myproposition}[矩母函数的性质]
        \begin{itemize}
            \item 当矩母函数存在时,它唯一地确定了$X$的分布.
            \item \begin{equation}
                E\left[X^n\right]=g^{(n)}(0),\quad n\geqslant 1.
            \end{equation}
            \item 对于\textbf{相互独立}的随机变量$X,Y$,
            \begin{equation}
                g_{X+Y}(t)=g_X(t)g_Y(t).
            \end{equation}
        \end{itemize}
    \end{myproposition}
\end{frame}

\begin{frame}
    \frametitle{矩母函数}
    \begin{block}{注}
        随机变量的矩母函数\textbf{不一定存在},所以
        现在也常用特征函数
        \begin{equation*}
            \varphi(t)=E\left[\mathrm e^{\mathrm itX}\right]
        \end{equation*}
        代替矩母函数. 
    \end{block}
    \begin{exampleblock}{随机和}
        记$X_1,X_2,\cdots$为一列\textbf{独立同分布}的随机变量,
        $N$为非负整数值随机变量,且与$X_i$序列\textbf{独立},
        则
        \begin{equation*}
            Y=\sum_{i=1}^NX_i
        \end{equation*}
        称为随机和. 


    \end{exampleblock}
\end{frame}

\begin{frame}
    \frametitle{随机和}
    \begin{exampleblock}{随机和的矩母函数及期望等数字特征}
        先算条件期望,再借助全期望公式即可求得
        随机和$Y$的矩母函数为
        \begin{equation}
            g_Y(t)=E\left[(g_X(t))^N\right],
        \end{equation}
        且由此可得出
        \begin{align}\label{eq:random_sum}
            EY&=EN\cdot EX,\\
            EY^2&=EN\cdot\mathrm{Var}(X)+EN^2\cdot E^2X,\\
            \mathrm{Var}(Y)&=EN\cdot \mathrm{Var}(X)+E^2X\cdot\mathrm{Var}(N).
        \end{align}
    \end{exampleblock}
\end{frame}

\subsection{其它}

\begin{frame}
    \frametitle{$\Gamma$函数}

    \begin{mydefinition}[$\Gamma$函数]
        对任意实数$x>0$,
        \begin{equation}
            \Gamma(x)=\int_0^{+\infty}t^{x-1}\mathrm e^{-t}\ \mathrm dt.
        \end{equation}
    \end{mydefinition}

    \begin{myproposition}[$\Gamma$函数的递推公式]
        当$x>0$时,有$\Gamma(x+1)=x\Gamma(x)$. 特别地,对$n\in\mathbb{N}$由
        \begin{align*}
            \Gamma(1)=1,\quad
            \Gamma(1/2)=\sqrt\pi
        \end{align*}
        得到
        \begin{align}
            \Gamma(n+1)=n!,\quad
            \Gamma(n+1/2)=\frac{(2n-1)!!}{2^n}\sqrt\pi.
        \end{align}
    \end{myproposition}
\end{frame}

\begin{frame}
    \frametitle{利用$\Gamma$函数求积分}
    \begin{exampleblock}{利用$\Gamma$函数求积分}
        \begin{align}
            \int_0^{+\infty}t^\alpha \mathrm e^{-\beta t}\ \mathrm dt&=\frac{\Gamma(\alpha+1)}{\beta^{\alpha+1}},\\
            \int_0^{+\infty}t^\alpha \mathrm e^{-\beta t^2}\ \mathrm dt&=\frac{\Gamma((\alpha+1)/2)}{2\beta^{(\alpha+1)/2}}.
        \end{align}
    \end{exampleblock}
\end{frame}