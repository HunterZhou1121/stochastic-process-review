\section{平稳过程}

% this section is written only based on the textbook, since the slides are flawed. 

\subsection{平稳过程的定义}

\begin{frame}
    \frametitle{严平稳过程}
    \begin{mydefinition}[严平稳过程]
        见定义\ref{def:strictly-stationary-process}.
    \end{mydefinition}
    \begin{myproposition}[严平稳过程的数字特征]
        设$X=\set{X(t),t\in T}$为严平稳过程,则
        \begin{itemize}
            \item 如果均值函数$m(t)=EX(t)$\textbf{存在},则必为常数,记为$m$.
            \item 如果方差函数$\mathrm{Var}(X(t))=E\parentheses{X(t)-m}^2$\textbf{存在},则它也是常数,记为$\sigma^2$. 
            \item 协方差函数仅与时间差有关,而与起点无关,记为
            $
                R(h)=E\brackets{(X(h)-m)(X(0)-m)}.
            $
            特别,由定义,$\mathrm{Var}(X(t))=R(0)$. 
            \item 自相关函数 $r(\tau)=E\brackets{X(t)X(t+\tau)}$只与时间差$\tau$有关而与起点$t$无关;将
            $\rho(v)=R(v)/\sigma^2=R(v)/R(0)$称为标准自相关函数. 由概率论中相关系数的性质\footnotemark 得$\rho(0)=1,\abs{\rho(v)}\leqslant1$.
        \end{itemize}
    \end{myproposition}
    \footnotetext{缪柏其, 张伟平. 概率论与数理统计. 高等教育出版社, 2022.}
\end{frame}

\setcounter{footnote}{0}

\begin{frame}
    \frametitle{(宽)平稳过程}
    \begin{mydefinition}[实(宽)平稳过程]\label{def:real-wide-stationary-process}
        见定义\ref{def:wide-stationary-process}. 注意这里的随机过程$X=\set{X(t),t\in T}$是实值随机过程. 
    \end{mydefinition}
    \begin{mydefinition}[复(宽)平稳过程]\label{def:complex-wide-stationary-process}
        设$X=\set{X(t),t\in T}$为一\textbf{复值}随机过程,如果对$\forall t\in T$,满足
        \begin{itemize}
            \item $EX^2(t)<\infty$;
            \item $EX(t)=m$;
            \item $R_X(s,t)=E\brackets{(X(t)-m)\overline{(X(s)-m)}}$仅与时间差$t-s$有关,
        \end{itemize}
        则称$X$为复(宽)平稳随机过程.
    \end{mydefinition}
    \begin{alertblock}{注意}
        定义\ref{def:complex-wide-stationary-process}与定义\ref{def:real-wide-stationary-process}的重要区别在于
        协方差函数的计算中增加了取复共轭的操作.
    \end{alertblock}
\end{frame}

\begin{frame}
    \frametitle{严平稳过程与宽平稳过程之间的关系}
    回顾第\ref{frame:wide-stationary-strict-stationary}页. 
    \begin{columns}
        \begin{column}{.6\columnwidth}
            \begin{alertblock}{注意}
                一般来说,这两个过程是互不包含的.
                \begin{itemize}
                    \item 严平稳过程由于不一定有二阶矩而不一定是宽平稳的;
                    \item 宽平稳过程由于其有限维联合分布可能不满足式(\ref{eq:strictly-stationary})
                            而不一定是严平稳的. 
                \end{itemize}
            \end{alertblock}    
        \end{column}
        \begin{column}{.3\columnwidth}
            \begin{exampleblock}{特例}
                \begin{itemize}
                    \item 如果严平稳过程的二阶矩存在,则它是宽平稳的;
                    \item 如果宽平稳过程是Gauss过程,则它是严平稳的. 
                \end{itemize}
            \end{exampleblock}  
        \end{column}        
    \end{columns}

    \begin{mydefinition}[Gauss过程]
        设$G=\set{G(t),-\infty<t<+\infty}$为一随机过程,
        如果对任意正整数$k$以及$k$个时刻$t_1\leqslant t_2\leqslant \cdots t_k$,
        $(G(t_1), G(t_2),\cdots, G(t_k))$的联合分布为$k$维正态分布,则称$G$为Gauss过程. 
    \end{mydefinition}

\end{frame}

\subsection{平稳过程的遍历性定理}

\begin{frame}
    \frametitle{平稳过程(序列)的遍历性}
    \begin{mydefinition}[均值遍历性]
        设$X=\set{X(t),-\infty<t<+\infty}$为一平稳过程(序列),若
        \begin{equation*}
            \overline X=\lim_{T\to\infty}\frac1{2T}\int_{-T}^{T}X(t)\ \mathrm dt\overset{L_2}{=}m
        \end{equation*}
        或
        \begin{equation*}
            \overline X=\lim_{N\to\infty}\frac1{2N+1}\sum_{k=-N}^NX(k)\overset{L_2}{=}m,
        \end{equation*}
        则称$X$的均值具有遍历性\footnotemark . 
    \end{mydefinition}
    如果随机过程(序列)的均值和协方差函数都具有遍历性,称此随机过程有遍历性(各态历经性). 
    \footnotetext{均值遍历性定义中的极限是均方极限,不是通常的极限;协方差函数遍历性的定义同理,略.}
\end{frame}

\setcounter{footnote}{0}

\begin{frame}
    \frametitle{均值遍历性定理}
    \begin{columns}
        \begin{column}{0.55\columnwidth}
            \begin{mytheorem}[均值遍历性定理]
                \begin{enumerate}
                    \item 设$X=\set{X_n,n=0,\pm1,\cdots}$为平稳序列,其协方差函数为$R(\tau)$,则$X$有均值遍历性的充分必要条件是
                    \begin{equation}
                        \lim_{N\to\infty}\frac1N\sum_{\tau=0}^{N-1}R(\tau)=0.
                    \end{equation}
                    \item 若$X=\set{X(t),-\infty<t<+\infty}$为平稳过程,则$X$有均值遍历性的充分必要条件是
                    \begin{equation}
                        \lim_{T\to\infty}\frac1T\int_0^{2T}\left(1-\frac{\tau}{2T}\right)R(\tau)\ \mathrm d\tau=0.
                    \end{equation}
                \end{enumerate}
            \end{mytheorem}            
        \end{column}
        \begin{column}{0.40\columnwidth}
            \begin{mycorollary}[平稳过程均值遍历性成立的充分条件]
                \begin{enumerate}
                    \item 对平稳序列,若
                    \begin{equation}
                        \displaystyle\lim_{\tau\to\infty}R(\tau)=0,
                    \end{equation}
                    则均值遍历性成立;
                    \item 对平稳过程,若
                    \begin{equation}
                        \displaystyle\int_{-\infty}^{+\infty}|R(\tau)|\ \mathrm d\tau<\infty,
                    \end{equation}
                    则均值遍历性成立.
                \end{enumerate}
            \end{mycorollary}          
        \end{column}        
    \end{columns}


\end{frame}

\begin{frame}
    \frametitle{协方差函数的遍历性}
    协方差函数的遍历性涉及到过程的四阶矩,一般很难验证. 但,对Gauss过程,判定容易得多. 
    \begin{mytheorem}[Gauss序列协方差函数的遍历性]
        设$X=\set{X_n,n=0,\pm1,\cdots}$是均值为0的Gauss平稳序列,如果
        \begin{equation}
            \lim_{N\to\infty}\frac1N\sum_{k=0}^{N-1}R^2(k)=0,
        \end{equation}
        则该Gauss序列的协方差函数有遍历性. 
    \end{mytheorem}
\end{frame}

\subsection{平稳过程的协方差函数和功率谱密度}

\begin{frame}
    \frametitle{平稳过程的协方差函数}
    % \begin{columns}
    %     \begin{column}{0.45\columnwidth}
    %         \begin{mydefinition}[平稳过程的导数]
    %             如果存在$Y(t),t\in T$使得
    %             \begin{equation}
    %                 \small\lim_{h\to0}E\abs{\frac{X(t+h)-X(t)}{h}-Y(t)}^2=0,
    %             \end{equation}
    %             则称$Y=\set{Y(t)}$为过程$X$在$t$点的均方导数,简称导数,并记为$X'(t)$或$\dfrac{\mathrm dX(t)}{\mathrm dt}$.
    %         \end{mydefinition}            
    %     \end{column}
    %     \begin{column}{0.45\columnwidth}
    %         \begin{myproposition}
    %             均方导数存在的充分必要条件是
    %             \begin{equation}
    %                 \small\lim_{\substack{h\to0\\k\to0}}\frac{R(0)-R(h)-R(k)+R(h-k)}{hk}
    %             \end{equation}
    %             存在且有限.
    %         \end{myproposition}            
    %     \end{column}        
    % \end{columns}
    \small
    \begin{mydefinition}[平稳过程的导数]
        如果存在$Y(t),t\in T$使得
        \begin{equation}
            \lim_{h\to0}E\abs{\frac{X(t+h)-X(t)}{h}-Y(t)}^2=0,
        \end{equation}
        则称$Y=\set{Y(t)}$为过程$X$在$t$点的均方导数,简称导数,并记为$X'(t)$或$\dfrac{\mathrm dX(t)}{\mathrm dt}$.
    \end{mydefinition} 
    \begin{myproposition}
        均方导数存在的充分必要条件是
        \begin{equation}
            \lim_{\substack{h\to0\\k\to0}}\frac{R(0)-R(h)-R(k)+R(h-k)}{hk}
        \end{equation}
        存在且有限.
    \end{myproposition}
\end{frame}

\begin{frame}
    \frametitle{平稳过程的协方差函数}
    \begin{myproposition}[平稳过程协方差函数的性质]
        对平稳过程$X$的协方差函数$R(\tau)$,有如下性质\footnotemark :
        \begin{enumerate}
            \item 对称性:$R(-\tau)=R(\tau)$. 
            \item 有界性:$\abs{R(\tau)}\leqslant R(0)$.
            \item 非负定性:对任意的时刻$t_m$和实数$a_n,n=1,2,\cdots,N$,有
                \begin{equation}
                    \sum_{n=1}^N\sum_{m=1}^Na_na_mR(t_n-t_m)\geqslant 0.
                \end{equation}
            \item 平稳过程$n$阶导数的协方差函数为
                \begin{equation}
                    \mathrm{Cov}\parentheses{X^{(n)}(t),X^{(n)}(t+\tau)}=(-1)^nR^{(2n)}(\tau). 
                \end{equation}
        \end{enumerate}
    \end{myproposition}
    \footnotetext{前三条性质是一般协方差函数即有的性质;第四条性质中假定涉及到的导数都存在. }
\end{frame}

\setcounter{footnote}{0}

\begin{frame}
    \frametitle{平稳过程的功率谱密度}
    \begin{mytheorem}[Wiener-Khintchine公式]\footnotesize
        假定$EX(t)=0$,且$\displaystyle\int_{-\infty}^{+\infty}\abs{R(\tau)}\ \mathrm d\tau<\infty$,则
        \begin{align}
            S(\omega)&=\int_{-\infty}^{+\infty}R(\tau)\mathrm e^{-\mathrm i\omega\tau}\ \mathrm d\tau,\\
            R(\tau)&=\frac1{2\pi}\int_{-\infty}^{+\infty}S(\omega)\mathrm e^{\mathrm i\omega\tau}\ \mathrm d\omega.
        \end{align}
    \end{mytheorem}
    \begin{mycorollary}\footnotesize
        由$R(\tau)$和$S(\omega)$是偶函数得到
        \begin{align}
            S(\omega)&=2\int_0^{+\infty}R(\tau)\cos\omega\tau\ \mathrm d\tau,\\
            R(\tau)&=\frac1{\pi}\int_0^{+\infty}S(\omega)\cos\omega\tau\ \mathrm d\omega. 
        \end{align}
    \end{mycorollary}
\end{frame}

\begin{frame}
    \frametitle{平稳过程的功率谱密度}
    \begin{myproposition}[功率谱密度的必要条件]
        \begin{itemize}
            \item 功率谱密度$S(\omega)$是$\omega$的非负实值偶函数. 
            \item 最常见的为有理谱密度
            \begin{equation*}
                S(\omega)=\frac{P(\omega)}{Q(\omega)}=s_0\frac{\omega^{2n}+a_{2n-2}\omega^{2n-2}+\cdots+a_2\omega^2+a_0}{\omega^{2m}+b_{2m-2}\omega^{2m-2}+\cdots+b_2\omega^2+b_0},
            \end{equation*}
            其中$s_0>0$,分母不能有实根且分母多项式次数至少比分子高2(也就是$m>n$). 
        \end{itemize}
    \end{myproposition}
\end{frame}

\begin{frame}
    \frametitle{由有理功率谱密度求协方差函数}
    \begin{exampleblock}{由有理功率谱密度求协方差函数}
        设平稳过程$X$的功率谱密度为有理谱密度
        \begin{equation*}
            S(\omega)=\frac{P(\omega)}{Q(\omega)}.
        \end{equation*}
        考虑$\tau\geqslant 0$的情形:设$a_1,a_2,\cdots,a_n$是$f(z)$在\textbf{上半平面的全部奇点},则根据命题 \ref{prop:residue-P/Q-exp},有
        \small
        \begin{align}
            R(\tau)&=\frac1{2\pi}\int_{-\infty}^{+\infty}S(\omega)\mathrm e^{\mathrm i\omega\tau}\ \mathrm d\omega\notag\\
            &=\frac1{2\pi}\cdot 2\pi\mathrm i\sum_{k=1}^n\Res{\frac{P(z)}{Q(z)}\mathrm e^{\mathrm i\tau z}}{a_k}\notag\\
            &=\mathrm i\sum_{k=1}^n\Res{\frac{P(z)\mathrm e^{\mathrm i\tau z}}{Q(z)}}{a_k}.
        \end{align}
    \end{exampleblock}
\end{frame}

\begin{frame}
    \frametitle{由有理功率谱密度求协方差函数(Cont.)}
    \begin{exampleblock}{由有理功率谱密度求协方差函数(Cont.)}
        根据推论 \ref{cor:residue-P/Q},有
        \begin{align}
            R(\tau)&=\mathrm i\sum_{k=1}^n\Res{\frac{P(z)\mathrm e^{\mathrm i\tau z}}{Q(z)}}{a_k}\notag\\
            &=\mathrm i\sum_{k=1}^n\frac{P(a_k)\mathrm e^{\mathrm i\tau a_k}}{Q'(a_k)}\\
            &=f(\tau),\quad \tau\geqslant 0\notag.
        \end{align}
        因为$R(\tau)$是关于$\tau$的偶函数,所以有
        \begin{equation}
            R(\tau)=f\parentheses{\abs{\tau}},\quad -\infty<\tau<+\infty. 
        \end{equation}
    \end{exampleblock}
\end{frame}