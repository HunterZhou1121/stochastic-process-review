%!TEX program = xelatex
\documentclass[aspectratio=169]{ctexbeamer}
\usepackage{amsmath, amssymb, amsfonts, mathtools, bm, siunitx}
\usepackage{tikz}
\usepackage{fontspec}
\usepackage{xeCJK}
\usepackage[UKenglish]{datetime2}
\usepackage{adjustbox}
\usepackage{amstext}
\usefonttheme{serif}
\usetheme{boadilla}

\title[SP Review]{Stochastic Process Review}
\author[Kryze, S.]{Satine Kryze}
\institute[USTC]{University of Science and Technology of China}
\date{\DTMtoday}

% 调整定义环境的样式
\theoremstyle{definition}
\setbeamertemplate{theorems}[numbered]
\newtheorem{mytheorem}{定理}[section]
\newtheorem{mydefinition}[mytheorem]{定义}
\newtheorem{mylemma}[mytheorem]{引理}
\newtheorem{mycorollary}[mytheorem]{推论}
\newtheorem{myexample}[mytheorem]{例}
\newtheorem{myproposition}[mytheorem]{命题}
% \newtheorem{axiom}[theorem]{公理}
\newtheorem{myaxiom}{公理}

\numberwithin{equation}{section}

\providecommand{\set}[1]{\left\{#1\right\}}
\providecommand{\parentheses}[1]{\left(#1\right)}
\providecommand{\brackets}[1]{\left[#1\right]}
\providecommand{\abs}[1]{\left\lvert#1\right\rvert}
\providecommand{\Res}[2]{\mathrm{Res}\brackets{#1, #2}}
\providecommand{\N}{\mathbb{N}}
\providecommand{\Z}{\mathbb{Z}}
\providecommand{\Q}{\mathbb{Q}}
\providecommand{\R}{\mathbb{R}}


\begin{document}

\maketitle

\begin{frame}
  \frametitle{大纲}
  \tableofcontents[hideallsubsections]
\end{frame}

\AtBeginSection[]{
\setbeamertemplate{footline}[footlineoff]
  \begin{frame}
    \frametitle{大纲}
    \tableofcontents[currentsection,subsectionstyle=show/show/hide]
  \end{frame}
\setbeamertemplate{footline}[footlineon]
}

\AtBeginSubsection[]{
\setbeamertemplate{footline}[footlineoff]
  \begin{frame}
    \frametitle{大纲}
    \tableofcontents[currentsection,subsectionstyle=show/shaded/hide]
  \end{frame}
\setbeamertemplate{footline}[footlineon]
}

\section{预备知识}

\subsection{随机过程的概率分布}

\begin{frame}
  \frametitle{有限维分布和数字特征}
  对于随机过程$\{X(t),t\in T\}$,过程的一维分布为
  $$
    F_t(x)=P(X(t)\leqslant x),
  $$
  过程的\textbf{一维均值函数}为
  $$
    \mu_X(t)=EX(t),
  $$
  过程的\textbf{方差函数}为
  $$
    \sigma_X^2(t)=\mathrm{Var}(X(t)).
  $$
\end{frame}

\begin{frame}
  \frametitle{有限维分布和数字特征}
  过程在$t_1,t_2$两个不同时刻值的联合二维分布为
  $$
  F_{t_1,t_2}(x_1,x_2)=P(X(t_1)\leqslant x_1,X(t_2)\leqslant x_2),
  $$
  过程的\textbf{自相关函数}为
  $$
  r_X(t_1,t_2)=E(X_1(t)X_2(t)),
  $$
  过程的\textbf{协方差函数}为
  $$
  R_X(t_1,t_2)\equiv\mathrm{Cov}(X(t_1),X(t_2))=E[(X(t_1)-\mu_X(t_1))(X(t_2)-\mu_X(t_2))].
  $$
\end{frame}

\begin{frame}
  \frametitle{自相关函数与协方差函数的性质}
  \begin{enumerate}
    \item \textbf{对称性}. 对任何$s,t$,有
          \begin{align*}
            r_X(s,t)&=r_X(t,s),\\
            R_X(s,t)&=R_X(t,s).
          \end{align*}

    \item \textbf{非负定性}. 对任何$t_1,t_2,\cdots,t_n\in T$及任意系数$b_1,b_2,\cdots,b_n$有
          \begin{align*}
            \sum_{i=1}^n\sum_{j=1}^nb_ib_jr_X(t_i,t_j)&\geqslant 0,\\
            \sum_{i=1}^n\sum_{j=1}^nb_ib_jR_X(t_i,t_j)&\geqslant 0.\\
          \end{align*}

  \end{enumerate}

\end{frame}

\subsection{平稳过程与独立增量过程}

\begin{frame}
    \frametitle{严平稳过程}
    \begin{mydefinition}[同分布的随机向量]
        如果一个随机向量$\bm X=(X_1,X_2,\cdots,X_n)$与
        另一个随机向量$\bm Y=(Y_1,Y_2,\cdots,Y_n)$有
        相同的联合分布函数,则称这两个随机向量是同分布的,
        记为
        $$
          \bm X\overset{d}{=}\bm Y.
        $$        
    \end{mydefinition}

    \begin{mydefinition}[严平稳过程]\label{def:strictly-stationary-process}
        如果随机过程$X(t)$对任意的$t_1,\cdots,t_n\in T$
        和任何$h$都有
        \begin{equation}\label{eq:strictly-stationary}
            (X(t_1+h),\cdots,X(t_n+h))\overset{d}{=} (X(t_1),\cdots,X(t_n)),            
        \end{equation}

        则称$X(t)$为\textbf{严格平稳的}(strictly stationary or strict-sense stationary). 
    \end{mydefinition}

\end{frame}

\begin{frame}
    \frametitle{(宽)平稳过程}
    \begin{mydefinition}[宽平稳过程]\label{def:wide-stationary-process}
        如果随机过程$X(t)$的所有二阶矩存在,并且$EX(t)=m$为常数,
        协方差函数$R_X(s,t)$只与时间差$t-s$有关,
        则称$X(t)$为\textbf{(宽)平稳的}(wide-sense stationary)或\textbf{二阶矩平稳的}. 
    \end{mydefinition}
    对宽平稳过程$X(t)$,由于对$-\infty<s,t<+\infty$,
    $$
        R_X(s,t)=R_X(0,t-s),
    $$
    所以可以记为$R_X(t-s)$. 显然对所有$t$,都有$R_X(t)=R_X(-t)$,
    即$R_X(t)$为偶函数. 

\end{frame}

\begin{frame}\label{frame:wide-stationary-strict-stationary}
    \frametitle{宽平稳过程与严平稳过程之间的关系}
    \begin{alertblock}{注意}
        一般来说,这两个过程是互不包含的.

        \begin{itemize}
            \item 严平稳过程由于不一定有二阶矩而不一定是宽平稳的;
            \item 宽平稳过程由于其有限维联合分布可能不满足式(\ref{eq:strictly-stationary})
                    而不一定是严平稳的. 
        \end{itemize}
        
        
    \end{alertblock}
    \begin{exampleblock}{特例}
        \begin{itemize}
            \item 如果严平稳过程的二阶矩存在,则它是宽平稳的;
            \item 如果宽平稳过程是Gauss过程,则它是严平稳的. 
        \end{itemize}
    \end{exampleblock}
\end{frame}

\begin{frame}
    \frametitle{独立增量过程}
    \begin{mydefinition}[独立增量过程与平稳独立增量过程]\label{def:independent-increment-process}
        对任意的$t_1<t_2<\cdots<t_n$且
        $t_1,\cdots,t_n\in T$,如果
        随机变量
        $$
            X(t_2)-X(t_1),X(t_3)-X(t_2),\cdots,X(t_n)-X(t_{n-1})
        $$
        是相互独立的,则称$X(t)$为\textbf{独立增量过程}.
        
        如果进一步有,对任意的$t_1,t_2$,
        \begin{equation*}
            X(t_1+h)-X(t_1)\overset{d}{=}X(t_2+h)-X(t_2),
        \end{equation*}
        则称$X(t)$为\textbf{平稳独立增量过程}. 
    \end{mydefinition}
    \begin{myproposition}
        可以证明,平稳独立增量过程的均值函数一定是$t$的线性函数. 
    \end{myproposition}
\end{frame}

\subsection{条件期望与矩母函数}

\begin{frame}
    \frametitle{条件期望}
    条件期望通常统一记为
    \begin{equation*}
        E(X|Y=y)=\int x\ \mathrm dF(x|y).
    \end{equation*}
    \begin{block}{注}
        $E(X|Y=y)=g(y)$ 是一个关于$y$的数值,
        而$E(X|Y)=g(Y)$ 是一个关于$Y$的随机变量函数,从而是随机变量.  
    \end{block}
\end{frame}

\begin{frame}
    \frametitle{条件期望的性质}
    \begin{myproposition}
        \begin{itemize}
            \item 若$X$与$Y$独立,则 $E(X|Y=y)=EX$.
            \item 条件期望的平滑性(全期望公式):
                \begin{equation}
                    E(E(X|Y))=EX.
                \end{equation}
            \item 对随机变量$X,Y$的函数$\phi(X,Y)$,有
                \begin{equation}
                    E[\phi(X,Y)|Y=y]=E[\phi(X,y)|Y=y].
                \end{equation}
        \end{itemize}
    \end{myproposition}
\end{frame}

\begin{frame}
    \frametitle{条件期望的性质}
    \begin{myproposition}
        若$X,Y$为随机变量,$EX,EY,E(g(Y))$存在,则
        \begin{itemize}
            \item 当$X,Y$独立时,有$E(Y|X)=EY$.
            \item $E(g(X)Y|X)=g(X)E(Y|X)$.
            \item $E(c|X)=c$.
            \item $E(g(X)|X)=g(X)$. 
            \item $E[Y-E(Y|X)]^2\leqslant E[Y-g(X)]^2$.
        \end{itemize}
    \end{myproposition}   
    \begin{block}{注}
        在均方误差最小的准则下,若要找到基于$X$对$Y$的最佳预报函数$g(\cdot)$,
        也就是要求出$g$使得$E[Y-g(X)]^2$最小,
        则可以证明所求出的$g(x)$就是$E(Y|X=x)$,这也就是最后一条性质的意义. 
    \end{block}
\end{frame}

\begin{frame}
    \frametitle{条件方差}
    \begin{mydefinition}[条件方差]
        若$E\{[Y-E(Y|X)]^2|X\}$存在,则称其为随机变量$X$条件下随机变量$Y$的
        条件方差,记为$D(Y|X)$.
    \end{mydefinition}
    \begin{myproposition}[条件方差的性质]
        \begin{itemize}
            \item $D(Y|X)=E\left(Y^2|X\right)-[E(Y|X)]^2$.
            \item $D(X|Y)=E\left(X^2|Y\right)-[E(X|Y)]^2$.
        \end{itemize}
    \end{myproposition}
    \begin{myproposition}
        \begin{equation*}
            D(Y)=E(D(Y|X))+D(E(Y|X)).
        \end{equation*}
    \end{myproposition}
\end{frame}

\begin{frame}
    \frametitle{矩母函数}
    \begin{mydefinition}[矩母函数]
        随机变量$X$的矩母函数(moment generating function, MGF)定义为
        随机变量$\exp\{tX\}$的期望,记作
        $g(t)$,即
        \begin{equation}
            g(t)=E\left[\mathrm e^{tX}\right].
        \end{equation}
    \end{mydefinition}
    \begin{myproposition}[矩母函数的性质]
        \begin{itemize}
            \item 当矩母函数存在时,它唯一地确定了$X$的分布.
            \item \begin{equation}
                E\left[X^n\right]=g^{(n)}(0),\quad n\geqslant 1.
            \end{equation}
            \item 对于\textbf{相互独立}的随机变量$X,Y$,
            \begin{equation}
                g_{X+Y}(t)=g_X(t)g_Y(t).
            \end{equation}
        \end{itemize}
    \end{myproposition}
\end{frame}

\begin{frame}
    \frametitle{矩母函数}
    \begin{block}{注}
        随机变量的矩母函数\textbf{不一定存在},所以
        现在也常用特征函数
        \begin{equation*}
            \varphi(t)=E\left[\mathrm e^{\mathrm itX}\right]
        \end{equation*}
        代替矩母函数. 
    \end{block}
    \begin{exampleblock}{随机和}
        记$X_1,X_2,\cdots$为一列\textbf{独立同分布}的随机变量,
        $N$为非负整数值随机变量,且与$X_i$序列\textbf{独立},
        则
        \begin{equation*}
            Y=\sum_{i=1}^NX_i
        \end{equation*}
        称为随机和. 


    \end{exampleblock}
\end{frame}

\begin{frame}
    \frametitle{随机和}
    \begin{exampleblock}{随机和的矩母函数及期望等数字特征}
        先算条件期望,再借助全期望公式即可求得
        随机和$Y$的矩母函数为
        \begin{equation}
            g_Y(t)=E\left[(g_X(t))^N\right],
        \end{equation}
        且由此可得出
        \begin{align}\label{eq:random_sum}
            EY&=EN\cdot EX,\\
            EY^2&=EN\cdot\mathrm{Var}(X)+EN^2\cdot E^2X,\\
            \mathrm{Var}(Y)&=EN\cdot \mathrm{Var}(X)+E^2X\cdot\mathrm{Var}(N).
        \end{align}
    \end{exampleblock}
\end{frame}

\subsection{其它}

\begin{frame}
    \frametitle{$\Gamma$函数}

    \begin{mydefinition}[$\Gamma$函数]
        对任意实数$x>0$,
        \begin{equation}
            \Gamma(x)=\int_0^{+\infty}t^{x-1}\mathrm e^{-t}\ \mathrm dt.
        \end{equation}
    \end{mydefinition}

    \begin{myproposition}[$\Gamma$函数的递推公式]
        当$x>0$时,有$\Gamma(x+1)=x\Gamma(x)$. 特别地,对$n\in\mathbb{N}$由
        \begin{align*}
            \Gamma(1)=1,\quad
            \Gamma(1/2)=\sqrt\pi
        \end{align*}
        得到
        \begin{align}
            \Gamma(n+1)=n!,\quad
            \Gamma(n+1/2)=\frac{(2n-1)!!}{2^n}\sqrt\pi.
        \end{align}
    \end{myproposition}
\end{frame}

\begin{frame}
    \frametitle{利用$\Gamma$函数求积分}
    \begin{exampleblock}{利用$\Gamma$函数求积分}
        \begin{align}
            \int_0^{+\infty}t^\alpha \mathrm e^{-\beta t}\ \mathrm dt&=\frac{\Gamma(\alpha+1)}{\beta^{\alpha+1}},\\
            \int_0^{+\infty}t^\alpha \mathrm e^{-\beta t^2}\ \mathrm dt&=\frac{\Gamma((\alpha+1)/2)}{2\beta^{(\alpha+1)/2}}.
        \end{align}
    \end{exampleblock}
\end{frame}

\begin{frame}
    \frametitle{留数定理与极点的判定}
    \begin{mytheorem}[留数定理]
        设$f(z)$在闭路$C$上解析,在$C$内部除了$n$个孤立奇点$a_1,a_2,\cdots,a_n$外解析,则
        \begin{equation}
            \int_Cf(z)\ \mathrm dz=2\pi\mathrm i\sum_{k=1}^n\Res{f(z)}{a_k}.
        \end{equation}
    \end{mytheorem}
    \begin{mytheorem}[极点的判定]
        设$a$是$f(z)$的孤立奇点,$m\in\N^+$,则$a$是$f(z)$的$m$级极点,当且仅当$a$是$1/f(z)$的$m$级零点. 
    \end{mytheorem}
\end{frame}

\begin{frame}
    \frametitle{留数的计算}
    \footnotesize
    \begin{mytheorem}[留数的计算:$m$级极点]
        设$a$是$f(z)$的$m$级极点,则
        \begin{equation}
            \Res{f(z)}{a}=\frac1{(m-1)!}\lim_{z\to a}\frac{\mathrm d^{m-1}}{\mathrm dz^{m-1}}\set{(z-a)^mf(z)}. 
        \end{equation}
        特别,若$a$是$f(z)$的1级极点,则
        \begin{equation}
            \Res{f(z)}{a}=\lim_{z\to a}(z-a)f(z).
        \end{equation}
    \end{mytheorem}
    \begin{mycorollary}\label{cor:residue-P/Q}
        设$P(z),Q(z)$都在$a$点解析,$P(a)\neq0,Q(a)=0,Q'(a)\neq0$,则
        \begin{equation}
            \Res{\frac{P(z)}{Q(z)}}{z}=\frac{P(a)}{\displaystyle\lim_{z\to a}\dfrac{\mathrm dQ(z)}{\mathrm dz}}. 
        \end{equation}
    \end{mycorollary}
\end{frame}

\begin{frame}
    \frametitle{利用留数定理计算定积分}
    \begin{myproposition}[有理函数广义积分]
        设$R(x)=\dfrac{P(x)}{Q(x)},x\in\R$,$P(x),Q(x)$是$x$的多项式,$Q(x)$比$P(x)$的次数至少高2次,且$\forall x\in\R,Q(x)\neq 0$,则
        \begin{equation}
            \int_{-\infty}^{+\infty}\frac{P(x)}{Q(x)}\ \mathrm dx=2\pi\mathrm i\sum_{k=1}^n\Res{\frac{P(z)}{Q(z)}}{a_k},
        \end{equation}
        其中$a_1,a_2,\cdots,a_n$是$\dfrac{P(z)}{Q(z)}$在\textbf{上半平面}的全部奇点. 
    \end{myproposition}
\end{frame}

\begin{frame}
    \frametitle{利用留数定理计算定积分}
    \small
    \begin{myproposition}\label{prop:residue-P/Q-exp}
        设$m>0$,$P(x),Q(x)$是实数$x$的多项式,且$Q(x)$比$P(x)$的次数至少高1次、$\forall x\in\R,Q(x)\neq 0$,则
        \begin{equation}
            I=\int_{-\infty}^{+\infty}\frac{P(x)}{Q(x)}\mathrm e^{\mathrm imx}\ \mathrm dx=2\pi\mathrm i\sum_{k=1}^n\Res{\frac{P(z)}{Q(z)}\mathrm e^{\mathrm imz}}{a_k},
        \end{equation}
        其中$a_1,a_2,\cdots,a_n$是$f(z)=\dfrac{P(z)}{Q(z)}\mathrm e^{\mathrm imz}$在上半平面的全部奇点. 
        特别,
        \begin{align}
            I_1&=\int_{-\infty}^{+\infty}\frac{P(x)}{Q(x)}\cos mx\ \mathrm dx=\mathrm{Re} I,\\
            I_2&=\int_{-\infty}^{+\infty}\frac{P(x)}{Q(x)}\sin mx\ \mathrm dx=\mathrm{Im} I.
        \end{align}        
    \end{myproposition}
\end{frame}

\section{Poisson过程}

\subsection{基本概念}

\begin{frame}
    \frametitle{Poisson分布}
    \begin{mydefinition}[Poisson分布]
        如果随机变量$X$的分布律为
        \begin{equation}
            P(X=k)=\frac{\lambda^k}{k!}\mathrm e^{-\lambda},\quad k=1,2,\cdots,\lambda>0,
        \end{equation}
        则称$X$服从参数为$\lambda$的\textbf{Poisson分布},记为$X\sim P(\lambda)$.
    \end{mydefinition}

    \begin{myproposition}[Poisson分布的数字特征]
        如果$X\sim P(\lambda)$,那么$EX=\lambda,\mathrm{Var}(X)=\lambda$. 
    \end{myproposition}
\end{frame}

\begin{frame}
    \frametitle{Poisson逼近定理}
    \begin{mytheorem}[Poisson逼近定理]
        如果 $X_n\sim B(n,p_n)$,且$n\to\infty,np_n\to\lambda>0$,那么
        \begin{equation}
            \lim_{n\to\infty}P(X_n=k)=\lim_{n\to\infty}\binom nk p_n^k(1-p_n)^{n-k}=\frac{\lambda^k}{k!}\mathrm e^{-\lambda},\quad k=0,1,2,\cdots.
        \end{equation}
    \end{mytheorem}
    \begin{exampleblock}{实际应用\footnotemark}
        \begin{itemize}
            \item $n\geqslant 30,np_n\leqslant 5$时即可应用;
            \item 当$n\geqslant100$时,$np_n\leqslant 10$的情形仍有较高的精度. 
        \end{itemize}
    \end{exampleblock}
    \footnotetext{缪柏其, 张伟平. 概率论与数理统计. 高等教育出版社, 2022}
\end{frame}

\setcounter{footnote}{0}

\begin{frame}
    \frametitle{计数过程}
    \begin{mydefinition}[计数过程]
        设$X(t)$表示直到$t$时刻为止,某事件$A$所出现的次数. 
        如果$X(t)$是取非负整数的随机变量,则称$X=\{X(t)\},t\geqslant 0$
        是\textbf{计数过程}. 
    \end{mydefinition}
    \begin{myproposition}[计数过程的性质\footnotemark]
        \begin{itemize}
            \item $X(t)\geqslant0$;
            \item $X(t)$取非负整数值;
            \item 若$s<t$,则$X(s)\leqslant X(t)$;
            \item 若$s<t$,则$X(t)-X(s)$表示在时间区间$[s,t]$内某事件$A$出现的次数. 
        \end{itemize}
    \end{myproposition}
    \footnotetext{一般还设$X(0)=0$.}
\end{frame}

\begin{frame}
    \frametitle{独立增量过程与平稳增量过程}
    \begin{mydefinition}[平稳增量过程]
        如果计数过程在任一区间内发生的事件个数只
        依赖于时间区间的长度,即计数过程$X(t)$
        在$(t,t+s](s>0)$内事件$A$发生的次数$X(t+s)-X(s)$
        仅与时间差$s$有关而与$t$无关,则
        计数过程$X(t)$称为\textbf{平稳增量过程}. 
    \end{mydefinition}
    \begin{mydefinition}[独立增量过程]
        见定义\ref{def:independent-increment-process}.
    \end{mydefinition}
\end{frame}

\begin{frame}
    \frametitle{Poisson过程}
    \begin{mydefinition}[Poisson过程的定义1]
        如果一个计数过程$X=\{X(t),t\geqslant0\}$满足条件:
        \begin{enumerate}
            \item $X(0)=0$;
            \item $X$是独立增量过程;
            \item 在任一长度为$t$的区间中,事件$A$发生的次数服从
            均值为$\lambda t$的Poisson分布,即
            $\forall s,t\geqslant 0$,
            \begin{equation}
                P(X(t+s)-X(s)=n)=\frac{(\lambda t)^n}{n!}\mathrm e^{-\lambda t},\quad n=0,1,\cdots,
            \end{equation}
        \end{enumerate}
        则称$X$是具有参数为$\lambda>0$的\textbf{Poisson过程}.
    \end{mydefinition}
    \begin{block}{注}
        由$EX(t)=\lambda t$可知$\lambda=EX(t)/t$,即单位时间内事件$A$发生
        平均次数,因此也称$\lambda$为此过程的\textbf{速度}或\textbf{强度}.         
    \end{block}
\end{frame}

\begin{frame}
    \frametitle{Poisson过程}
    \begin{mydefinition}[Poisson过程的定义2]
        如果一个计数过程$X=\{X(t),t\geqslant 0\}$满足条件:
        \begin{enumerate}
            \item $X(0)=0$;
            \item $X$是独立平稳增量过程;
            \item \begin{equation}\label{eq:poisson-process-definition-2-3}
                P(X(t+h)-X(t)=1)=\lambda h+o(h),\quad h>0;
            \end{equation}
            \item \begin{equation}\label{eq:poisson-process-definition-2-4}
                P(X(t+h)-X(t)\geqslant 2)=o(h),
            \end{equation}
            则称$X$是参数为$\lambda >0$的Poisson过程. 
        \end{enumerate}
    \end{mydefinition}
    \begin{block}{注}
        条件(\ref{eq:poisson-process-definition-2-3})和(\ref{eq:poisson-process-definition-2-4})说明,
        在充分小的时间间隔内,最多有一个事件发生. 这种假设对于很多物理现象较容易得到满足. 
    \end{block}
\end{frame}

\subsection{Poisson过程的性质}

\begin{frame}
    \frametitle{Poisson过程的数字特征}
    \begin{myproposition}[Poisson过程的数字特征]
        设$\{X(t),t\geqslant 0\}$是强度为$\lambda$的Poisson过程,则
        \begin{itemize}
            \item 均值$m_X(t)=EX(t)=E[X(t)-X(0)]=\lambda t$;
            \item 方差$D_X(t)=\mathrm{Var}(X(t))=\lambda t$;
            \item 自相关函数$r_X(s,t)=E[X(s)X(t)]=\lambda s(\lambda t+1),\quad s<t$;
            \item 协方差函数$R_X(s,t)=r_X(s,t)-EX(s)EX(t)=\lambda s,\quad s<t$;
            \item 特征函数$g_X(u)=\exp\{\lambda t(\mathrm e^{\mathrm iu}-1)\}$.
        \end{itemize}
    \end{myproposition}
\end{frame}

\begin{frame}
    \frametitle{时间间隔与等待时间}
    \begin{mydefinition}[时间间隔与等待时间]
        设$\{X(t),t\geqslant 0\}$是参数为$\lambda$的Poisson过程.
        令$X(t)$表示$t$时刻事件$A$发生的次数,则
        第$n-1$次与第$n$次事件间的\textbf{时间间隔}记作$T_n$,而
        \begin{equation*}
            W_n=\sum_{i=1}^nT_i
        \end{equation*}
        称为第$n$次事件的\textbf{到达时间}或\textbf{等待时间}. 
    \end{mydefinition}
    \begin{mytheorem}[时间间隔的分布]
        设$\{X(t),t\geqslant 0\}$ 是具有参数$\lambda$的Poisson过程,
        $\{T_n(n\geqslant 1)\}$是对应的时间间隔序列,则随机变量$T_n(n=1,2,\cdots)$
        是相互独立的,并且都服从均值为$1/\lambda$的指数分布,即$T_n\overset{\mathrm{i.i.d.}}{\sim} \mathrm{Exp}(\lambda)$.
    \end{mytheorem}
\end{frame}

\begin{frame}
    \frametitle{时间间隔与等待时间}
    \begin{block}{注}
        平稳独立增量的假定保证了过程的无记忆性,因此
        指数间隔是意料之中的.
    \end{block}

    \begin{mytheorem}[等待时间的分布]
        设$\{W_n(n\geqslant 1)\}$是与Poisson过程$\{X(t),t\geqslant 0\}$
        对应的一个等待时间序列,则$W_n$服从参数为$n,\lambda$的$\Gamma$分布或
        Erlang分布,记作$W_n\sim\Gamma(n,\lambda)$. 它的概率密度为
        \begin{equation}
            f_{W_n}(t)=
            \begin{cases}
                \lambda \mathrm e^{-\lambda t}\dfrac{(\lambda t)^{n-1}}{(n-1)!},&t\geqslant 0,\\
                0,&t<0.
            \end{cases}
        \end{equation}
    \end{mytheorem}
\end{frame}

\begin{frame}
    \frametitle{时间间隔与等待时间:例}
    \begin{exampleblock}{例}
        设$\{X_1(t),t\geqslant0\}$和$\{X_2(t),t\geqslant 0\}$
        是两个相互独立的Poisson过程,它们的强度分别为$\lambda_1$和
        $\lambda_2$. 即$W_k^{(1)}$为过程$X_1(t)$的第$k$次事件到达时间,
        $W_1^{(2)}$为过程$X_2(t)$的第$1$次事件到达时间,则第一个Poisson过程
        的第$k$次事件发生比第二个Poisson过程的第$1$次事件发生更早的概率为
        \begin{align}
            P\left(W_k^{(1)}<W_1^{(2)}\right)
            &=\iint\limits_{x<y}f(x,y)\ \mathrm dx\mathrm dy\nonumber\\
            &=\iint\limits_{x<y}f_{W_k^{(1)}}(x)f_{W_1^{(2)}}(y)\ \mathrm dx\mathrm dy\nonumber\\
            &=\left(\frac{\lambda_1}{\lambda_1+\lambda_2}\right)^k.
        \end{align}
    \end{exampleblock}
\end{frame}

\begin{frame}
    \frametitle{等待时间的条件分布}
    \begin{mytheorem}[等待时间的条件分布]
        设$\{X(t),t\geqslant 0\}$是Poisson过程,若
        已知在$[0,t]$内事件$A$发生$n$次,则
        这$n$个等待时间(到达时间)$W_1,W_2,\cdots,W_n$
        与相应于$n$个$[0,t]$上均匀分布的独立随机变量的
        顺序统计量有相同的分布.

        此时$W_1,W_2,\cdots,W_n$在已知$X(t)=n$的条件下的
        条件概率密度为
        \begin{equation}
            f_{W_1,W_2,\cdots,W_n|X(t)=n}(t_1,t_2,\cdots,t_n|n)=\frac{n!}{t^n},\quad 0<t_1<t_2<\cdots<t_n\leqslant t.
        \end{equation}
    \end{mytheorem}
\end{frame}

\begin{frame}
    \frametitle{等待时间的条件分布:例}
    \begin{exampleblock}{例}
        顾客到达车站的过程是速率为$\lambda$的Poisson过程.
        若火车在时刻$t$离站,问在$(0,t]$区间内
        顾客的平均总等待时间是多少?
    \end{exampleblock}
    第$i$位到达的顾客的到达时间为$W_i$,等到时刻$t$发车需要等待
    $t-W_i$. 在$(0,t]$区间内共来了$N(t)$位顾客,所以总等待时间为
    \begin{small}
    \begin{equation*}
        \sum_{i=1}^{N(t)}(t-W_i),
    \end{equation*}
    \end{small}
    而所求的平均总等待时间就是它的期望. 为求它,可以先求条件期望.
    \begin{small}
    \begin{align*}
        E\left[\left.\sum_{i=1}^{N(t)}(t-W_i)\right|N(t)=n\right]
        =E\left[\left.\sum_{i=1}^{n}(t-W_i)\right|N(t)=n\right]
        =nt-E\left[\left.\sum_{i=1}^nW_i\right|N(t)=n\right].
    \end{align*}
    \end{small}
\end{frame}

\begin{frame}
    \frametitle{等待时间的条件分布:例(Cont.)}
    注意到给定$N(t)=n$,$W_i,i=1,2,\cdots,n$的联合密度与$(0,t]$
    上均匀分布中随机样本$U_i,i=1,2,\cdots,n$的次序统计量$U_{(i)},i=1,2,\cdots,n$
    的联合密度是一样的. 于是,
    \begin{small}
        \begin{equation*}
            E\left[\left.\sum_{i=1}^nW_i\right|N(t)=n\right]
            =E\left[\sum_{i=1}^n U_{(i)}\right]
            =E\left[\sum_{i=1}^n U_{i}\right]
            =\frac{nt}2.
        \end{equation*}        
    \end{small}
    因此
    \begin{small}
        \begin{equation*}
            E\left[\left.\sum_{i=1}^{N(t)}(t-W_i)\right|N(t)=n\right]=nt-\frac{nt}2=\frac{nt}2.
        \end{equation*}        
    \end{small}
    最后得到
    \begin{small}
        \begin{equation*}
            E\left[\sum_{i=1}^{N(t)}(t-W_i)\right]
            =E\left[E\left[\left.\sum_{i=1}^{N(t)}(t-W_i)\right|N(t)\right]\right]
            =E\left[\frac{N(t)t}2\right]
            =\frac t2EN(t)=\frac{\lambda t^2}{2}.
        \end{equation*}
    \end{small}
\end{frame}

\begin{frame}
    \frametitle{剩余寿命与年龄}
    \begin{mydefinition}[剩余寿命与年龄]
        设$X(t)$为在$(0,t]$内事件$A$发生的次数,
        $W_n$表示第$n$个事件发生的时刻,则
        $W_{X(t)}$表示在$t$时刻前最后一个事件发生的时刻,
        $W_{x(t)+1}$表示在$t$时刻后首次事件发生的时刻.
        令
        \begin{equation*}
            \begin{cases}
                S(t)=W_{X(t)+1}-t,\\
                V(t)=t-W_{X(t)}.
            \end{cases}
        \end{equation*}
        称$S(t)$为\textbf{剩余寿命}或\textbf{剩余时间},
        $V(t)$为\textbf{年龄}. 
    \end{mydefinition}
    由定义可知,
    \begin{equation*}
        \forall t\geqslant 0,S(t)\geqslant 0,
        0\leqslant V(t)\leqslant t.
    \end{equation*}
\end{frame}

\begin{frame}
    \frametitle{剩余寿命与年龄}
    \begin{mytheorem}[剩余寿命与年龄的分布]
        设$\{X(t),t\geqslant 0\}$是具有参数$\lambda$的Poisson过程,则
        \begin{itemize}
            \item $S(t)$与$\{T_n,n\geqslant 1\}$同分布. 即,
            \begin{equation}
                P(S(t)\leqslant x)=1-\mathrm e^{-\lambda x},\quad x\geqslant 0.
            \end{equation}
            \item $V(t)$的分布为“截尾”的指数分布. 即,
            \begin{equation}
                P(V(t)\leqslant x)=
                \begin{cases}
                    1-\mathrm e^{-\lambda x},&0\leqslant x<t,\\
                    1,&x\geqslant t.
                \end{cases}
            \end{equation}
        \end{itemize}
    \end{mytheorem}
\end{frame}

\subsection{Poisson过程的推广}

\begin{frame}
    \frametitle{非齐次Poisson过程}
    \begin{mydefinition}[非齐次Poisson过程]
        如果一个计数过程$X=\{X(t),t\geqslant 0\}$满足条件:
        \begin{enumerate}
            \item $X(0)=0$;
            \item $X$是独立增量过程;
            \item \begin{equation}
                P(X(t+h)-X(t)=1)=\lambda(t)h+o(h),\quad h>0;
            \end{equation}
            \item \begin{equation}
                P(X(t+h)-X(t)\geqslant 2)=o(h),
            \end{equation}
        \end{enumerate}
        则称$X(t)$是参数为$\lambda(t)$的非齐次Poisson过程. 
    \end{mydefinition}

\end{frame}

\begin{frame}
    \frametitle{非齐次Poisson过程}
    \begin{myproposition}[非齐次Poisson过程的分布]
        该非齐次Poisson过程的增量$X(t+h)-X(t)$的分布为
        \begin{equation}
            P(X(t+h)-X(t)=k)
            =\frac{\left(\displaystyle\int_t^{t+h}\lambda u\ \mathrm du\right)^k\exp\left(-\displaystyle\int_t^{t+h}\lambda(u)\ \mathrm du\right)}{k!},
            \quad k=0,1,\cdots.
        \end{equation}
    \end{myproposition}
    \begin{exampleblock}{简记}
        设$s<t$,若记
        \begin{equation*}
            m(s,t)=\int_s^t\lambda(u)\ \mathrm du
        \end{equation*}
        为累计强度函数,则
        \begin{equation*}
            X(t)-X(s)\sim P(m(s,t)).
        \end{equation*}
    \end{exampleblock}
\end{frame}

\begin{frame}
    \frametitle{复合Poisson过程}
    如果事件的发生依从一Poisson过程,而
    每一次事件都附带一个随机变量(如费用、损失等),
    则累计值过程
    \begin{equation*}
        X(t)=\sum_{i=1}^{N(t)}Y_i
    \end{equation*}
    在$N(t)$是参数为$\lambda$的Poisson时是一个
    复合Poisson过程(compound Poisson process),其中$Y_i$为独立同分布的随机变量,
    它们有均值$EY_i=\mu$,方差$\mathrm{Var}(Y_i)=\tau^2$.
    \begin{myproposition}[复合Poisson的数字特征]
        $X(t)$是随机和. 
        由式(\ref{eq:random_sum})等可立即得到
        \begin{align}
            EX(t)&=\lambda\mu t,\\
            \mathrm{Var}(X(t))&=\lambda(\tau^2+\mu^2)t.
        \end{align}
    \end{myproposition}
\end{frame}

\begin{frame}
    \frametitle{更新过程}
    \begin{mydefinition}[更新过程]
        如果$X_i,i=1,2,\cdots,$为一串非负的随机变量,
        它们独立同分布,分布函数为$F(x)$,
        记 $W_0=0$,$\displaystyle W_n=\sum_{i=1}^nX_i$
        表示第$n$次事件发生的时刻,则称
        \begin{equation}
            N(t)=\max\{n:W_n\leqslant t\}
        \end{equation}
        为更新过程. 
    \end{mydefinition}
    这里$N(t)$表示到时刻$t$时事件的总数. $W_i,i=1,2,\cdots$也常常
    称为更新点,在这些更新点上过程又重新开始. 
    
    在更新过程中事件平均发生的次数称为\textbf{更新函数},记作
    \begin{equation*}
        m(t)=EN(t). 
    \end{equation*}
\end{frame}

\begin{frame}
    \frametitle{更新过程}

    \begin{myproposition}[更新过程的分布与更新函数]
        更新过程$N(t)$的分布为
        \begin{equation*}
            P(N(t)=n)=F^{(n)}(t)-F^{(n+1)}(t),
        \end{equation*}
        而更新函数为
        \begin{equation*}
            m(t)=\sum_{n=1}^\infty F^{(n)}(t),
        \end{equation*}
        其中$F^{(n)}(t)$为$F(t)$的$n$重卷积,$F(t)$是
        $X_i$的分布函数. 
    \end{myproposition}
\end{frame}

\begin{frame}
    \frametitle{更新过程}
    \begin{mytheorem}[基本更新定理]
        对一个一般的更新过程$N(t)$,若事件间隔时间$X_i$的期望为$\mu$,即
        $EX_i=\mu$,则
        \begin{equation}
            \lim_{t\to\infty}\frac{EN(t)}{t}=\frac1\mu.
        \end{equation}
        也就是说,单位时间内的平均事件次数当$t$趋于无穷时
        趋于一个固定的极限$1/\mu$. 
    \end{mytheorem}
    \begin{myproposition}
        \begin{itemize}
            \item Poisson过程是更新过程的特殊情形. 即,当更新过程
            的事件时间间隔$X_i$为独立同指数分布时,更新过程就是Poisson过程. 
            \item Poisson过程是一类非常特殊的连续时间Markov过程——纯生过程的特例. 
        \end{itemize}
    \end{myproposition}
\end{frame}

\section{Markov链}

\subsection{Markov链的定义及转移概率}

\begin{frame}
    \frametitle{Markov链的定义}
    \begin{mydefinition}[Markov链]
        设有随机过程$\{X_n,n\in T\}$,若对于任意的正整数$n\in T$
        和任意的状态$i_0,i_1,\cdots,i_n,i_{n+1}\in I$,条件概率满足Markov性质(无后效性)
        \begin{equation}
            P(X_{n+1}=i_{n+1}|X_0=i_0,X_1=i_1,\cdots,X_n=i_n)=P(X_{n+1}=i_{n+1}|X_n=i_n),
        \end{equation}
        就称$\{X_n,n\in T\}$为\textbf{Markov链},简称\textbf{马氏链}. 
    \end{mydefinition}
\end{frame}

\begin{frame}
    \frametitle{Markov链的转移概率}
    \begin{mydefinition}[Markov链的转移概率]
        称条件概率
        \begin{equation}
            p_{ij}(n)=P(X_{n+1}=j|X_n=i),\quad i,j\in I
        \end{equation}
        为Markov链$\{X_n,n\in T\}$在时刻$n$的\textbf{一步转移概率},简称为\textbf{转移概率}. 
        
    \end{mydefinition}
    \begin{mydefinition}[齐次Markov链与平稳转移概率]
        若对任意的$i,j\in I$,Markov链$\{X_n,n\in T\}$
        的转移概率$p_{ij}(n)$与时刻$n$无关,则称Markov链是
        \textbf{齐次}的,并记$p_{ij}(n)$为$p_ij$. 此时,我们也说Markov
        链具有\textbf{平稳转移概率}. 
    \end{mydefinition}
\end{frame}

\begin{frame}
    \frametitle{一步转移概率矩阵}
    \begin{mydefinition}
        \begin{equation}
            \bm P=
            \begin{pmatrix}
                p_{00} & p_{01} & \cdots & p_{0n} & \cdots\\
                p_{11} & p_{11} & \cdots & p_{1n} & \cdots\\
                \vdots & \vdots & \ddots & \vdots & \vdots\\
            \end{pmatrix}
        \end{equation}
        称为是Markov链$\{X_n,n\in T\}$的\textbf{一步转移概率矩阵}.
    \end{mydefinition}
    \begin{myproposition}[一步转移概率矩阵的性质]
        \begin{itemize}
            \item $p_{ij}\geqslant 0,\quad i,j\in I$;
            \item $\displaystyle\sum_{j\in I}p_{ij}=1,\quad i\in I$.
        \end{itemize}
    \end{myproposition}
\end{frame}

\begin{frame}
    \frametitle{$n$步转移概率}
    \begin{mydefinition}[$n$步转移概率]
        称条件概率
        \begin{equation}
            p_{ij}^{(n)}=P(X_{m+n}=j|X_m=i),\quad i,j\in I,m\geqslant 0,n\geqslant 1
        \end{equation}
        为Markov链$\{X_n,n\in T\}$的\textbf{$n$步转移概率},并称
        \begin{equation}
            \bm P^{(n)}=
            \begin{pmatrix}
                p_{ij}^{(n)}
            \end{pmatrix}
        \end{equation}
        为 Markov 链的\textbf{$n$步转移概率矩阵}.
        规定:
        \begin{equation}
            p_{ij}^{(0)}=
            \begin{cases}
                0,&i\neq j,\\
                1,&i=j.
            \end{cases}
        \end{equation}
    \end{mydefinition}
\end{frame}

\begin{frame}
    \frametitle{$n$步转移概率的性质}
    \begin{mytheorem}[$n$步转移概率的性质]
        设$\{X_n,n\in T\}$为Markov链,则对于任意整数
        $n\geqslant 0,0\leqslant l<n$和任意状态$i,j\in I$,
        $n$步转移概率$p_{ij}^{(n)}$具有下列性质:
        \begin{small}
        \begin{enumerate}
            \item (Chapman-Kolmogorov方程)
            \begin{equation}
                p_{ij}^{(n)}=\sum_{k\in I}p_{ik}^{(l)}p_{kj}^{(n-l)};
            \end{equation}
            \item \begin{equation*}
                p_{ij}^{(n)}=\sum_{k_1\in I}\cdots\sum_{k_{n-1}\in I}p_{ik_1}p_{k_1k_2}\cdots p_{k_{n-1}j};
            \end{equation*}
            \item \begin{equation}
                \bm P^{(n)}=\bm P\cdot \bm P^{(n-1)};
            \end{equation}
            \item \begin{equation}
                \bm P^{(n)}=\bm P^n.
            \end{equation}
        \end{enumerate}
        \end{small}
    \end{mytheorem}
\end{frame}

\begin{frame}
    \frametitle{初始概率和绝对概率}
    \begin{columns}
        \begin{column}{.45\linewidth}
            \begin{mydefinition}
                \begin{itemize}
                    \item 初始概率:
                    \begin{equation*}
                        p_j=P(X_0=j),\quad j\in I;
                    \end{equation*}
                    \item 初始分布:
                    \begin{equation*}
                        \{p_j\}=\{p_j,\quad j\in I\};
                    \end{equation*}
                    \item 初始概率向量:
                    \begin{equation*}
                        \bm P^T(0)=(p_1,p_2,\cdots).
                    \end{equation*}
                \end{itemize}
            \end{mydefinition}
        \end{column}
        \begin{column}{.45\linewidth}
            \begin{mydefinition}
                \begin{itemize}
                    \item 绝对概率:
                    \begin{equation*}
                        p_j(n)=P(X_n=j),\quad j\in I;
                    \end{equation*}
                    \item 绝对分布:
                    \begin{equation*}
                        \{p_j(n)\}=\{p_j(n),\quad j\in I\};
                    \end{equation*}
                    \item 绝对概率向量:
                    \begin{equation*}
                        \bm P^T(n)=(p_1(n),p_2(n),\cdots),\quad n>0.
                    \end{equation*}
                \end{itemize}
            \end{mydefinition}
        \end{column}        
    \end{columns}
\end{frame}

\begin{frame}
    \frametitle{绝对概率的性质}

    \begin{mytheorem}[绝对概率的性质]
        设$\{X_n,n\in T\}$为Markov链,则对于任意整数$n\geqslant 1$和任意状态
        $j\in I$,绝对概率$p_j(n)$具有下列性质:
        \begin{enumerate}
            \item \begin{equation*}
                p_j(n)=\sum_{i\in I}p_i p_{ij}^{(n)};
            \end{equation*}
            \item \begin{equation*}
                p_j(n)=\sum_{i\in I}p_i(n-1)p_{ij};
            \end{equation*}
            \item \begin{equation*}
                \bm P^T(n)=\bm P^T(0)\cdot \bm P^{(n)};
            \end{equation*}
            \item \begin{equation*}
                \bm P^T(n)=\bm P^T(n-1)\cdot \bm P.
            \end{equation*}
        \end{enumerate}
    \end{mytheorem}
\end{frame}

\subsection{Markov链的状态分类}

\begin{frame}
    \frametitle{Markov链的状态分类}
    设$\{X_n,n>0\}$是齐次Markov链,其状态空间$I=\{0,1,2,\cdots\}$,转移概率是$p_{ij},\quad i,j\in I$,
    初始分布为$\{p_j,\quad j\in I\}$.
    以下是将要讨论的Markov链的状态分类以及状态之间的关系:
    \begin{itemize}
        \item 状态间的可达关系与互达关系;
        \item 状态的周期与非周期性;
        \item 状态的常返性与瞬过性(非常返性). 
    \end{itemize}
\end{frame}

\begin{frame}
    \frametitle{可达关系与互达关系}

    \begin{mydefinition}[可达关系与互达关系]
        \begin{enumerate}
            \item 若存在$n>0$使得$p_{ij}^{(n)}>0$,则称自状态$i$\textbf{可达}(accessible)状态$j$,并
            记为$i\to j$;
            \item 若$i\to j$且$j\to i$,则称状态$i$与状态$j$\textbf{互达}(communicate),并记为$i\leftrightarrow j$. 
        \end{enumerate}
    \end{mydefinition}

    \begin{columns}
        \begin{column}{.45\linewidth}
            \begin{mytheorem}[可达关系与互达关系的传递性]
                \begin{itemize}
                    \item 若$i\to j$且$j\to k$,则$i\to k$;
                    \item 若$i\leftrightarrow j$且$j\leftrightarrow k$,则$i\leftrightarrow k$.
                \end{itemize}
            \end{mytheorem}            
        \end{column}
        \begin{column}{.45\linewidth}
            \begin{mytheorem}[互达状态的等价性]
                互达关系是等价关系;有互达关系的状态是同一类型的,即若$i\leftrightarrow j$,则
                \begin{enumerate}
                    \item $i,j$同为常返或瞬过(非常返);
                    \item $i,j$同为正常返或零常返;
                    \item $i,j$有相同的周期. 
                \end{enumerate}
            \end{mytheorem}            
        \end{column}
    \end{columns}
\end{frame}

\begin{frame}
    \frametitle{状态的周期性}
    \begin{mydefinition}[状态的周期性]
        若集合$\left\{n:n\geqslant 1,p_{ii}^{(n)}>0\right\}\neq\varnothing$,则称该集合
        的最大公约数
        \begin{equation}
            d=d(i)=\gcd\left\{n:p_{ii}^{(n)}>0\right\}
        \end{equation}
        为状态$i$的\textbf{周期}.
        如果$d>1$,称状态$i$为\textbf{周期}的;如果$d=1$,称状态
        $i$为\textbf{非周期}的.
    \end{mydefinition}
    \begin{mytheorem}
        如果状态$i$的周期为$d$,则存在正整数$M$使得对一切$n\geqslant M$有
        $p_{ii}^{(nd)}>0$. 
    \end{mytheorem}
\end{frame}

\begin{frame}
    \frametitle{状态的常返性}
    \begin{mydefinition}[首达概率]
        状态$i$经$n$步首次到达状态$j$的概率
        \begin{align*}
            f_{ij}^{(n)}&=P(X_{m+n}=j,X_{m+v}\neq j,1\leqslant v\leqslant n-1|X_m=i),\quad n\geqslant 1
        \end{align*}
        称为\textbf{首达概率}. 规定$f_{ij}^{(0)}=0$. 
        系统从状态$i$出发,经有限步迟早会(首次)到达状态$j$的概率为
        \begin{equation*}
            f_{ij}=\sum_{n=1}^\infty f_{ij}^{(n)},\quad 0\leqslant f_{ij}^{(n)}\leqslant f_{ij}\leqslant 1.
        \end{equation*}
    \end{mydefinition}
    \begin{mydefinition}[平均返回时间]
        称期望值$\displaystyle\mu_i=\sum_{n=1}^\infty n\cdot f_{ii}^{(n)}$
        为状态$i$的\textbf{平均返回时间}. 
    \end{mydefinition}
\end{frame}

\begin{frame}
    \frametitle{状态的常返性}
    \begin{mydefinition}[状态的常返性与瞬过性]
        \begin{itemize}
            \item 若$f_{ii}=1$,则称状态$i$\textbf{常返}(recurrent);
            \item 若$f_{ii}<1$,则称状态$i$\textbf{瞬过}(transient)或\textbf{非常返}.
        \end{itemize}
    \end{mydefinition}
    \begin{mydefinition}[常返态的分类]
        \begin{itemize}
            \item 若$\mu_i<\infty$,则称常返态$i$是\textbf{正常返}的;
            \item 若$\mu_i=\infty$,则称常返态$i$是\textbf{零常返}的. 
        \end{itemize}
    \end{mydefinition}
    \begin{mydefinition}[遍历态]
        非周期的正常返态称为\textbf{遍历的}(ergodic).
    \end{mydefinition}
\end{frame}

\begin{frame}
    \frametitle{$n$步转移概率与首达概率的关系}
    \begin{mytheorem}[$n$步转移概率与首达概率的关系]
        对任意状态$i,j\in I$及$1\leqslant n<\infty$,有
        \begin{equation*}
            p_{ij}^{(n)}=\sum_{k=1}^n f_{ij}^{(k)}p_{jj}^{(n-k)}=\sum_{k=0}^nf_{ij}^{(n-k)}p_{jj}^{(k)}.
        \end{equation*}
    \end{mytheorem}
    \begin{exampleblock}{应用}
        可以利用下面的变式来求从状态$i$经$n$步首次到达状态$j$的概率:
        \begin{equation}
            f_{ij}^{(n)}=p_{ij}^{(n)}-\sum_{k=1}^{n-1}f_{ij}^{(k)}p_{jj}^{(n-k)}.
        \end{equation}
    \end{exampleblock}
\end{frame}

\begin{frame}
    \frametitle{常返性的判别}
    \begin{mytheorem}[常返性的判别]
        \begin{enumerate}
            \item <1->状态$i$常返的充分必要条件为
            $\displaystyle\sum_{n=0}^\infty p_{ii}^{(n)}=\infty$;\\
            状态$i$非常返的充分必要条件为
            $\displaystyle\sum_{n=0}^\infty p_{ii}^{(n)}=\frac{1}{1-f_{ii}}<\infty$.
            \item <1->若状态$i$是常返态,则\\
                $i$是零常返当且仅当$\displaystyle\lim_{n\to\infty}p_{ii}^{(n)}=0$;\\
                $i$是遍历态当且仅当$\displaystyle\lim_{n\to\infty}p_{ii}^{(n)}=1/\mu_i>0$.
            \item <1->若$i$是周期为$d$的常返态,则$\displaystyle\lim_{n\to\infty}p_{ii}^{(nd)}=d/\mu_i$;\\
                若$i$是非常返态,则$\displaystyle\lim_{n\to\infty}p_{ii}^{(n)}=0$.
        \end{enumerate}
    \end{mytheorem}
\end{frame}

\subsection{Markov链状态空间的分解}

\begin{frame}
    \frametitle{基本概念}
    \begin{mydefinition}
        \begin{itemize}
            \item 状态空间$I$的子集$C$,若对于任意$i\in C$及$k\notin C$
            都有$p_{ik}=0$,则称子集$C$为(随机)\textbf{闭集}. 
            \item 若闭集$C$的状态互达,则称$C$为\textbf{不可约}的. 
            \item 若Markov链$\set{X_n,n\in T}$的状态空间$I$是不可约的,
            则称该Markov链为\textbf{不可约}(irreducible).
        \end{itemize}
    \end{mydefinition}
    \begin{mytheorem}
        $C$是闭集的充分必要条件是:对于任意$i\in C$及$k\notin C$,都有
        \begin{equation*}
            p_{ik}^{(n)}=0,\quad n\geqslant 1.
        \end{equation*}
        特别。单点集$C=\set{i}$是闭集当且仅当状态$i$是吸收态(即$p_{ii}=1$).
    \end{mytheorem}
\end{frame}

\begin{frame}
    \frametitle{Markov链状态空间的分解}
    \begin{mytheorem}
        任一Markov链的状态空间$I$,可唯一地分解成
        若干个互不相交的子集之和
        \begin{equation*}
            I=D\cup C_1\cup C_2\cup\cdots,
        \end{equation*}
        使得
        \begin{enumerate}
            \item $D$由全体非常返态组成;
            \item 每个$C_n$是常返态组成的不可约闭集;
            \item $C_n$中的状态同类(同为正常返或零常返)有相同的周期,且
            \begin{equation*}
                f_{jk}=1,\quad j,k\in C_n,
            \end{equation*}或者说$j\leftrightarrow k$. 
        \end{enumerate}
        称$C_n$为\textbf{基本常返闭集}.
    \end{mytheorem}
\end{frame}

\begin{frame}
    \frametitle{关于Markov链状态的几个结论}
    \begin{myproposition}
        \begin{itemize}
            \item 若Markov链有一个零常返态,则必有无穷多个零常返态;
            \item 有限状态的Markov链,不可能含有零常返态,也不能全是非常返态;
            \item 不可约的有限状态Markov链必为正常返;
            \item 直线上的对称随机游动是零常返的,非对称的随机游动是瞬过的,且二维对称随机游动也是零常返的,
            但三维及以上的对称随机游动却是瞬过的;
            \item 对一个不可约、非周期、有限状态Markov链,存在$N$使得当$n\geqslant N$时
            $n$步转移概率矩阵$\bm P^{(n)}$的所有元素都非零. 这样的Markov链称为是正则的. 
            对一个正则的有限状态Markov链,极限分布总是存在,且与初始分布无关. 即,
            \begin{equation*}
                \lim_{n\to\infty}p_{ij}^{(n)}=\pi_j,\quad i,j\in I.
            \end{equation*}
            这在下一节将详细讨论. 
        \end{itemize}
    \end{myproposition}
\end{frame}

\subsection{Markov链的极限定理与平稳分布}

\begin{frame}
    \frametitle{Markov链的极限定理}
    \begin{mytheorem}[Markov链的极限定理]
        Markov链的$n$步转移概率$p_{ij}^{(n)}$的极限为
        \begin{equation}
            \lim_{n\to\infty}p_{ij}^{(n)}=
            \begin{cases}
                0,&j\text{ 为瞬过(非常返)或零常返},\\
                1/\mu_j,&j\text{ 为遍历态(非周期的正常返)},\\
                \text{不确定},&j\text{ 为周期正常返}.
            \end{cases}
        \end{equation}
        但若状态$j$是周期为$d$的常返状态,则有
        \begin{equation}
            \lim_{n\to\infty}p_{jj}^{(nd)}=d/\mu_j.
        \end{equation}
    \end{mytheorem}
\end{frame}

\begin{frame}
    \frametitle{Markov链的遍历性}
    \begin{mydefinition}[Markov链的遍历性]
        设齐次Markov链$\set{X_n,n\geqslant 0}$的状态空间
        为$I$,若对于一切$i,j\in I$,$n$步转移概率$p_{ij}^{(n)}$
        存在不依赖于$i$的极限
        \begin{equation}
            \lim_{n\to\infty}p_{ij}^{(n)}=p_j(>0)=\frac1{\mu_j},
        \end{equation}
        则称该Markov链具有遍历性,并称$p_j$为状态$j$的稳态概率. 
    \end{mydefinition}
    \begin{block}{注}
        Markov链具有遍历性的充分必要条件是它不可约、非周期、正常返. 
    \end{block}
\end{frame}

\begin{frame}
    \frametitle{Markov链的平稳分布}
    \begin{mydefinition}[Markov链的平稳分布]
        称绝对概率分布$\set{\pi_j,\quad j\in I}$为Markov链的\textbf{平稳分布},如果它满足
        %\begin{small}
            \begin{equation}
                \begin{cases}
                    \pi_j=\displaystyle\sum_{i\in I}\pi_ip_{ij},\quad j\in I,\\
                    \displaystyle\sum_{i\in I}\pi_i=1\text{ 且 }\pi_j\geqslant 0,\quad j\in I.
                \end{cases}
            \end{equation}            
        %\end{small}
    \end{mydefinition}
    \begin{block}{注}
        若记概率分布 $\bm\pi^T=(\pi_1,\pi_2,\cdots)$,一步转移矩阵 $\bm P=\begin{pmatrix}
            p_{ij}
        \end{pmatrix}$,那么有
        %\begin{small}
            \begin{align}
                \bm\pi^T=\bm\pi^T\cdot\bm P\text{ 或者说 }
                \bm\pi=\bm P^T\cdot\bm\pi.
            \end{align}            
        %\end{small}
        意义:$\pi_j$与时间推移$n$无关. 在任意时刻,系统处于同一状态的概率是相同的.
    \end{block}
\end{frame}

\begin{frame}
    \frametitle{Markov链平稳分布的判别}
    \begin{mytheorem}[Markov链平稳分布的判别]
        不可约、非周期Markov链是正常返的充分必要条件是:存在
        平稳分布$\set{\pi_j,\quad j\in I}$,且此平稳分布就是极限分布
        $\set{1/\mu_j,\quad j\in I}$. 即,
        \begin{equation}
            \lim_{n\to\infty}p_{ij}^{(n)}=\frac1{\mu_j}=\pi_j.
        \end{equation} 
    \end{mytheorem}
    \begin{columns}
        \begin{column}{.18\columnwidth}
            \begin{mycorollary}
                不可约、非周期、有限状态的Markov链必存在平稳分布. 
            \end{mycorollary}
        \end{column}
        \begin{column}{.22\columnwidth}
            \begin{mycorollary}
                若不可约Markov链的所有状态是非常返或零常返的,则不存在平稳分布. 
            \end{mycorollary}
        \end{column}
        \begin{column}{.5\columnwidth}
            \begin{mycorollary}
                若$\set{\pi_j,\quad j\in I}$是不可约非周期Markov链的
                平稳分布,则
                \begin{equation}
                    \lim_{n\to\infty}p_j(n)=\frac1{\mu_j}=\pi_j.
                \end{equation}
            \end{mycorollary}
        \end{column}        
    \end{columns}
\end{frame}

\begin{frame}
    \frametitle{一般齐次Markov链的平稳分布}
    \textbf{齐次}Markov链是否存在平稳分布?如果存在,是否唯一?如何计算?
    我们分三种情况讨论. 
    \begin{mytheorem}[不可约遍历链]
        设$X=\set{X_n,n=0,1,\cdots}$是\textbf{不可约的遍历链},
        则$X$存在唯一的极限分布
        \begin{equation*}
            \set{\pi_j=\frac1{\mu_{jj}},\quad j\in I},
        \end{equation*}
        且此时的极限分布就是平稳分布. 

        平稳分布可以通过求解下列方程组求解:
        \begin{equation*}
            \begin{cases}
                \pi_j=\displaystyle\sum_{i\in I}\pi_i p_{ij},\quad j\in I,\\
                \displaystyle\sum_{i\in I}\pi_i=1.
            \end{cases}
        \end{equation*}
    \end{mytheorem}
\end{frame}

\begin{frame}
    \frametitle{一般齐次Markov链的平稳分布}
    \begin{mytheorem}[不可约正常返链]
        设$X=\set{X_n,n=0,1,\cdots}$是\textbf{不可约齐次Markov链},
        其状态空间$I$中的每个状态都是\textbf{正常返},
        则$X$有唯一的平稳分布
        \begin{equation*}
            \set{\pi_j=\frac1{\mu_{jj}},\quad j\in I}.
        \end{equation*}

        平稳分布可以通过求解下列方程组求解:
        \begin{equation*}
            \begin{cases}
                \pi_j=\displaystyle\sum_{i\in I}\pi_i p_{ij},\quad j\in I,\\
                \displaystyle\sum_{i\in I}\pi_i=1.
            \end{cases}
        \end{equation*}
    \end{mytheorem}
\end{frame}

\begin{frame}
    \frametitle{一般齐次Markov链的平稳分布}
    \begin{mytheorem}[一般齐次Markov链]
        设$X$的状态空间$S=D\cup C_0\cup C_1\cup\cdots$,其中$D$是
        非常返状态集,$C_0$是零常返状态集,$C_m,\ m=1,2,\cdots$
        是正常返状态的不可约闭集. 
        
        记$
            H=\displaystyle\bigcup_{k\geqslant i}C_k
        $,
        则\begin{enumerate}
            \item $X$不存在平稳分布的充分必要条件是$H=\varnothing$;
            \item $X$存在唯一平稳分布的充分必要条件是只有一个正常返的不可约闭集;
            \item $X$存在无穷多个平稳分布的充分必要条件是存在两个及以上正常返的不可约闭集. 
        \end{enumerate}
    \end{mytheorem}
    \begin{block}{注}
        若$X$存在两个及以上正常返的不可约闭集,则在每个闭集内分别求解平稳分布,
        再将这些平稳分布进行线性组合就得到了整个Markov链的平稳分布,但该平稳分布不唯一.
    \end{block}
\end{frame}

\section{平稳过程}

\section{Brown运动}

\begin{frame}
  \frametitle{致谢}
  \centerline{\Large 谢谢!}
\end{frame}

\end{document}
